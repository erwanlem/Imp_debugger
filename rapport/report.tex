\documentclass{article}
\usepackage[T1]{fontenc}

\usepackage[margin=4cm]{geometry}

\usepackage[french]{babel}
\usepackage{parskip}

\title{Travail d'étude et de recherche\\Débogueur pour langage impératif}

\author{Erwan Lemattre}
\date{Janvier 2024 -- Avril 2024}

\begin{document}



\maketitle

\vspace{1cm}
\section{Introduction}
Ce travail d'étude et de recherche a pour objectif de développer un débogueur 
pour langage impératif en utilisant le langage OCaml. Ce projet s'inscrit dans 
la continuation du cours de compilation dans lequel avait déjà été développé 
des interpréteurs pour langages impératifs et objets. Le langage 
utilisé pour ce débogueur se base en partie sur le langage l'un de ces langages.
C'est un langage impératif simple appelé \texttt{Imp} dont nous décrirons les 
spécificités dans la première partie de ce rapport. Ce langage a été la base 
sur laquelle a été construit toutes les fonctionnalités nécessaires à la 
création du débogueur. Nous discuterons en détails des fonctionnalités de 
débogage et de leurs implémentations dans la seconde partie de ce rapport.

\newpage


\section{Le langage \texttt{Imp}}

\subsection{Fonctionnalités}
Ce langage contient toutes les instructions de base qu'on peut attendre 
d'un langage impératif:
\begin{itemize}
    \item Boucle \texttt{while}
    \item Condition \texttt{if}/\texttt{else}
    \item Fonctions
    \item Tableaux de taille fixe
\end{itemize}


\subsection{Stockage des variables}
Le langage \texttt{Imp} permet de créer des fonctions, des variables globales ainsi 
que des variables globales. L'exécution pas à pas ainsi et les fonctionnalités comme 
le retour en arrière ont nécessité une structuration différente du code. En effet, 
le fait de pouvoir retourner en arrière implique le stockage des états du programme 
à chaque nouvelle instruction ou suite d'instruction. On peut retrouver dans le 
fichier \texttt{global.ml} l'ensemble des variables globales à l'interpréteur 
dont la pile \texttt{undo\_stack} qui est une liste des états du programme. À chaque 
nouvelle commande, l'état est ajouté à la pile. Pour revenir en arrière il suffit 
simplement de dépiler. On trouve également dans le fichier \texttt{gloabl.ml} les 
environnements locaux aux fonctions ainsi que l'environnement global. On utilise à 
présent des références qui peuvent facilement être modifiées. Cela permet lors d'un 
retour en arrière de récupérer l'ancien élément dans la pile et de le définir 
comme l'environnement courant.


\subsection{Inférence de type}
\subsubsection{Un première réflexion}
Le langage \texttt{Imp} n'a pas de définition de type explicite pour les variables 
et les fonctions. La première idée pour réaliser la vérification de type a été 
de vérifier seulement les variables et les fonctions dont on connait le type. 
Les éléments dont le type est inconnu sont accepté dans tous les cas car ils 
peuvent être du type souhaité (on saura à l'exécution si le type est le bon). 
Cette méthode implique plus d'erreur à l'exécution du programme.

\subsubsection{Vérifier avec de l'inférence de type}
La seconde idée, qui est celle utilisée dans ce projet, a été de réaliser un 
système d'inférence de type. L'idée de l'inférence de type est ici de 
générer des contraintes puis ensuite d'utiliser un algorithme d'unification 
pour vérifier que toutes les contraintes sont respectées. Les contraintes 
sont générées dans le fichier \texttt{typechecker.ml}.
Pour chaque expression on génère un nouveau type qui représente le type de 
l'expression. Les noms de variables sont générés de manière unique par 
la fonction \texttt{get\_var\_name}. La fonction \texttt{type\_expr} va 
retourner pour chaque expression la liste des contraintes sur cette 
expression ainsi que le type de retour de l'expression. Les instructions 
n'ayant quant à elles pas de valeur de retour, elles renvoient seulement 
les contraintes associées.

\subsubsection{Inférer le type des fonctions}
Une question intéressante à traiter a été comment inférer le type des 
fonctions. Il y a eu deux idées. D'abord, j'ai pensé à faire l'inférence 
sur les fonctions unes à unes et séparemment. Le problème est que cela 
implique une vérification des fonctions dans un ordre précis: il faut 
vérifier les fonctions qui sont appelées avant les fonctions dans 
lesquelles il y a les appels. Cela empêche également les fonctions 
récursives. Cette méthode étant trop restrictive, il a fallu faire 
autrement. La seconde idée utilisée pour ce langage est de regrouper 
toutes les contraintes ensembles et de les résoudres en une seule fois.
Avant la vérification des types on commence par générer un ensemble de 
variables pour les fonctions. Lors de la vérification il suffit de récupérer 
le nom de ces variables pour nos contraintes. Cette méthode ne permet pas le 
polymorphisme des fonctions et implique que chaque fonction est d'un seul 
et unique type.

\section{Débogage}

\subsection{Les fonctionnalités du débogueur}

\subsection{Suivi de la mémoire}

\subsection{Affichage du code}


\end{document}