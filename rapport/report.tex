\documentclass{article}
\usepackage[T1]{fontenc}

\usepackage[margin=4cm]{geometry}

\usepackage[french]{babel}
\usepackage{parskip}

\title{Travail d'étude et de recherche\\Débogueur pour langage impératif}

\author{Erwan Lemattre}
\date{Janvier 2024 -- Avril 2024}

\begin{document}



\maketitle

\vspace{1cm}
\section{Introduction}
Ce travail d'étude et de recherche a pour objectif de développer un débogueur 
pour langage impératif en utilisant le langage OCaml. Ce projet s'inscrit dans 
la continuation du cours de compilation dans lequel avait déjà été développé 
des interpréteurs pour langages impératifs et objets. Le langage 
utilisé pour ce débogueur se base en partie sur le langage l'un de ces langages.
C'est un langage impératif simple appelé \texttt{Imp} dont nous décrirons les 
spécificités dans la première partie de ce rapport. Ce langage a été la base 
sur laquelle a été construit toutes les fonctionnalités nécessaires à la 
création du débogueur. Nous discuterons en détails des fonctionnalités de 
débogage et de leurs implémentations dans la seconde partie de ce rapport.

\newpage


\section{Le langage \texttt{Imp}}

\subsection{Fonctionnalités}
Ce langage contient toutes les instructions de base qu'on peut attendre 
d'un langage impératif:
\begin{itemize}
    \item Boucle \texttt{while}
    \item Condition \texttt{if}/\texttt{else}
    \item Fonctions
    \item Tableaux de taille fixe
\end{itemize}


\subsection{Stockage des variables}
\label{store_var}
Le langage \texttt{Imp} permet de créer des fonctions, des variables globales ainsi 
que des variables globales. L'exécution pas à pas ainsi et les fonctionnalités comme 
le retour en arrière ont nécessité une structuration différente du code. En effet, 
le fait de pouvoir retourner en arrière implique le stockage des états du programme 
à chaque nouvelle instruction ou suite d'instruction. On peut retrouver dans le 
fichier \texttt{global.ml} l'ensemble des variables globales à l'interpréteur 
dont la pile \texttt{undo\_stack} qui est une liste des états du programme. À chaque 
nouvelle commande, l'état est ajouté à la pile. Pour revenir en arrière il suffit 
simplement de dépiler. On trouve également dans le fichier \texttt{gloabl.ml} les 
environnements locaux aux fonctions ainsi que l'environnement global. On utilise à 
présent des références qui peuvent facilement être modifiées. Cela permet lors d'un 
retour en arrière de récupérer l'ancien élément dans la pile et de le définir 
comme l'environnement courant.


\subsection{Inférence de type}
\subsubsection{Un première réflexion}
Le langage \texttt{Imp} n'a pas de définition de type explicite pour les variables 
et les fonctions. La première idée pour réaliser la vérification de type a été 
de vérifier seulement les variables et les fonctions dont on connait le type. 
Les éléments dont le type est inconnu sont accepté dans tous les cas car ils 
peuvent être du type souhaité (on saura à l'exécution si le type est le bon). 
Cette méthode implique plus d'erreur à l'exécution du programme.

\subsubsection{Vérifier avec de l'inférence de type}
La seconde idée, qui est celle utilisée dans ce projet, a été de réaliser un 
système d'inférence de type. L'idée de l'inférence de type est ici de 
générer des contraintes puis ensuite d'utiliser un algorithme d'unification 
pour vérifier que toutes les contraintes sont respectées. Les contraintes 
sont générées dans le fichier \texttt{typechecker.ml}.
Pour chaque expression on génère un nouveau type qui représente le type de 
l'expression. Les noms de variables sont générés de manière unique par 
la fonction \texttt{get\_var\_name}. La fonction \texttt{type\_expr} va 
retourner pour chaque expression la liste des contraintes sur cette 
expression ainsi que le type de retour de l'expression. Les instructions 
n'ayant quant à elles pas de valeur de retour, elles renvoient seulement 
les contraintes associées.

\subsubsection{Inférer le type des fonctions}
Une question intéressante à traiter a été comment inférer le type des 
fonctions. Il y a eu deux idées. D'abord, j'ai pensé à faire l'inférence 
sur les fonctions unes à unes et séparemment. Le problème est que cela 
implique une vérification des fonctions dans un ordre précis: il faut 
vérifier les fonctions qui sont appelées avant les fonctions dans 
lesquelles il y a les appels. Cela empêche également les fonctions 
récursives. Cette méthode étant trop restrictive, il a fallu faire 
autrement. La seconde idée utilisée pour ce langage est de regrouper 
toutes les contraintes ensembles et de les résoudres en une seule fois.
Avant la vérification des types on commence par générer un ensemble de 
variables pour les fonctions. Lors de la vérification il suffit de récupérer 
le nom de ces variables pour nos contraintes. Cette méthode ne permet pas le 
polymorphisme des fonctions et implique que chaque fonction est d'un seul 
et unique type.

\section{Débogage}

\subsection{Les fonctionnalités du débogueur}

\subsubsection{Commandes de débogage}
Les commandes proposées sont:
\begin{itemize}
    \item \texttt{next} : passer à l'instrution suivante
    \item \texttt{so}   : step over -- si appel de fonction, boucle ou condition passe au dessus
    \item \texttt{step} : s'arrête au prochain breakpoint. Si pas de breakpoint exécute tout le programme
    \item \texttt{break n} : ajoute un point d'arrêt à la ligne n
    \item \texttt{undo} : retourner en arrière
    \item \texttt{exit} : quitte le programme
\end{itemize}

Les commandes se basent principalement sur la fonction \verb|step| qui permet d'exécuter une 
unique instruction et retourne le nouvelle environnement et les instructions suivantes.
Le fait de retourner les instructions suivantes permet dans certains cas d'ajouter des 
instructions après l'exécution d'une instruction. Prenons par exemple le cas d'une instruction 
\verb|while|. Si la condition d'entrée est vraie alors on ajoute le corps de la boucle dans la 
liste des prochaines instructions. Sinon rien n'est ajouté et les prochaines instructions sont 
celles qui suivent le \verb|while|. On retrouve également ce fonctionnement avec les instructions 
conditionnelles. Nous verrons par la suite le cas particulier des appels de fonction.
À partir de \verb|step| on peut passer à l'instruction suivante (\verb|next|), sauter une boucle ou 
condition (\verb|so|) en répétant \verb|step| tant qu'on est dans la boucle et aller au prochain 
point d'arrêt (\verb|step|) en continuant \verb|step| tant que l'instruction n'est pas un point d'arrêt.
Les points d'arrêt sont stockés dans une table de hachage avec pour clé les numéros de ligne des points 
d'arrêt. Enfin comme expliqué en \ref{store_var} le retour en arrière se fait en récupérant le 
dernier état du programme.

\subsubsection{Les appels de fonctions}
Les appels de fonctions ont été un point délicat à traiter. Les appels de fonctions 
sont des expressions qui comme les boucles ou les conditions vont ajouter un ensemble 
d'instructions. De plus, il faut créer un nouvel environnement local à l'appel. Ces 
premières conditions n'ont pas vraiment posé de problème. La complexité vient du fait 
que des appels de fonctions peuvent se trouver par exemple en paramètre d'appels de 
fonctions, dans un tableau ou encore dans une opération arithmétique : il n'est 
plus possible d'exécuter une instruction en une seule fois. La solution a été de créer 
une fonction qui vérifie s'il y a un appel de fonction dans la prochaine instruction.
Dans le cas où il n'y en a pas, on peut exécuter l'instruction et passer à la suivante.
Dans le cas où on trouve un appel, on le remplace par une expression 
spéciale appelée \verb|continuite| et on rajoute l'instruction modifiée à la séquence 
d'instruction. On ajoute ensuite les instructions de notre fonction. Ainsi les 
prochaines instructions seront celle de la fonction. Lorsqu'on trouve une instruction 
\verb|return| la valeur retournée est ajouté dans une pile de retour. Un fois la 
fonction terminée on retrouve notre instruction dont l'appel a été remplacé par 
\verb|continuite|. L'expression \verb|continuite| est simplement remplacée par le 
retour de l'appel de fonction stocké dans la pile de retour. Ce mécanisme d'appel 
de fonction permet par exemple de gérer des cas comme \verb|a = f(x + g(x), h(y) - g(x))|
pour lesquels on doit rester à la même instruction après les quatres appels. 


\subsection{Suivi de la mémoire}
Dans l'objectif de pouvoir suivre l'évolution de la mémoire, j'ai ajouté 
les informations \textit{alive} et \textit{free} sur les tableaux. Ces 
informations permettent de savoir si un tableau est toujours utilisé ou si 
dans le cas contraire il n'est plus accessible et a été libéré. Pour obtenir 
ces informations j'ai ajouté un identifiant unique à chaque tableau. À partir des 
variables de notre environnement il suffit ensuite de marquer les identifiants des 
tableaux qu'on trouve lors du suivi des variables. Les identifiants non marqués sont 
libres. On retrouve ici un fonctionnement similaire à celui des \textit{Garbage Collector} 
avec le \textit{mark and sweep}.

\subsection{Affichage du code}
L'affichage du code se fait avec un \textit{pretty printer}. Ce \textit{pretty printer} nous 
permet d'obtenir une indentation du code pour obtenir un code lisible. Le 
problème soulevé par l'affichage a été de colorer seulement une instruction pour 
indiquer notre position dans le code. De plus la première version de ce débogueur 
utilisant la librairie \textit{ncurses}, il a fallu trouver une solution pour laisser 
la possibiliter d'ajouter de la couleur sans "casser" l'indentation. La solution 
a été d'ajouter des balises \verb|<color>| avec le \textit{pretty printer}. Les 
balises étant ajoutées avec la librairie \textit{Format} d'OCaml, cela nous garantit 
que l'indentation sera conservée. J'ai ensuite utilisé dans expressions régulières afin 
d'extraire les balises de la chaine de caractères obtenue. On peut finalement facilement 
ajouter de la couleur sur la chaine extraite.

\section{Conclusion}

\end{document}